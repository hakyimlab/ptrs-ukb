% latex table generated in R 3.6.2 by xtable 1.8-4 package
% Thu Nov  5 17:34:47 2020
\begin{table}[ht]
\centering
\begingroup\scriptsize
\begin{tabular}{llll}
  \toprule
{\textbf{UKB Field Description}} & {\textbf{UKB Field ID}} & {\textbf{Tag}} & {\textbf{Phenotype Category}} \\ 
  \midrule
Standing height & 50 & Height & Height \\ 
  Diastolic blood pressure, automated reading & 4079 & DBP & Blood pressures \\ 
  Systolic blood pressure, automated reading & 4080 & SBP & Blood pressures \\ 
  Body mass index (BMI) & 21001 & BMI & BMI \\ 
  White blood cell (leukocyte) count & 30000 & WBC & Blood cell counts \\ 
  Red blood cell (erythrocyte) count & 30010 & RBC & Blood cell counts \\ 
  Haemoglobin concentration & 30020 & Hb & Haemoglobin related \\ 
  Haematocrit percentage & 30030 & Ht & Haemoglobin related \\ 
  Mean corpuscular volume & 30040 & MCV & Haemoglobin related \\ 
  Mean corpuscular haemoglobin & 30050 & MCH & Haemoglobin related \\ 
  Mean corpuscular haemoglobin concentration & 30060 & MCHC & Haemoglobin related \\ 
  Platelet count & 30080 & Platelet & Blood cell counts \\ 
  Lymphocyte count & 30120 & Lymphocyte & Blood cell counts \\ 
  Monocyte count & 30130 & Monocyte & Blood cell counts \\ 
  Neutrophill count & 30140 & Neutrophil & Blood cell counts \\ 
  Eosinophill count & 30150 & Eosinophil & Blood cell counts \\ 
  Basophill count & 30160 & Basophil & Blood cell counts \\ 
   \bottomrule
\end{tabular}
\endgroup
\caption{\textbf{Meta information of the phenotypes retrieved from UK Biobank which were used in the analysis}. The ``Tag'' column shows the short name of the phenotyes used in this paper. And phenotypes are assigned into five categories which are shown in ``Phenotype Category'' column} 
\label{tab:trait_table}
\end{table}
